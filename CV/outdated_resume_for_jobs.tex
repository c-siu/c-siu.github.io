%%%%%%%%%%%%%%%%%%%%%%%%%%%%%%%%%%%%%%%%%
% Medium Length Professional CV
% LaTeX Template
% Version 2.0 (8/5/13)
%
% This template has been downloaded from:
% http://www.LaTeXTemplates.com
%
% Original author:
% Rishi Shah 
%
% Important note:
% This template requires the resume.cls file to be in the same directory as the
% .tex file. The resume.cls file provides the resume style used for structuring the
% document.
%
%%%%%%%%%%%%%%%%%%%%%%%%%%%%%%%%%%%%%%%%%

%----------------------------------------------------------------------------------------
%	PACKAGES AND OTHER DOCUMENT CONFIGURATIONS
%----------------------------------------------------------------------------------------

\documentclass{resume} % Use the custom resume.cls style

\usepackage{url}
\usepackage{hyperref}

\usepackage[left=0.75in,top=0.6in,right=0.75in,bottom=0.6in]{geometry} % Document margins
\newcommand{\tab}[1]{\hspace{.2667\textwidth}\rlap{#1}}
\newcommand{\itab}[1]{\hspace{0em}\rlap{#1}}

\name{Chunyin, SIU (Alex)} % Your name
\address{657 Rhodes Hall, Cornell University, Ithaca, NY 14853}
\address{
\url{cs2323@cornell.edu} \\
\url{https://c-siu.github.io}
}
% Your address
%\address{123 Pleasant Lane \\ City, State 12345} % Your secondary addess (optional)
%\address{} % Your phone number and email

\usepackage[T1]{fontenc}

\begin{document}

%----------------------------------------------------------------------------------------
%	EDUCATION SECTION
%----------------------------------------------------------------------------------------

\begin{rSection}{Summary}

\begin{itemize}
\item Data analyst specializing in non-Euclidean data (manifold learning, network analysis and topological data analysis) since 2017
\item Developed 2 algorithms, published 3 journal articles and 3 preprints, made 1 \href{https://www.youtube.com/watch?v=X5c3_bbwQ7Q}{YouTube video}
\item Collaborated with mathematicians, statisticians and computer scientists, and mentored 4 undergraduates
\item Embrace challenging data analysis tasks, and enjoy collaboration with people with diverse backgrounds and sharing our results with a wider audience
\end{itemize}

\end{rSection}

\begin{rSection}{Education}

{\bf Cornell University}, Ithaca, NY \hfill
{\em 2019 -- present} \\
PhD Applied Mathematics; supervised by Prof Gennady Samorodnitsky

{\bf The Chinese University of Hong Kong}, Hong Kong \hfill
{\em 2017 -- 2019} \\
MPhil Mathematics; supervised by Prof Ronald Lui %\\ \hfill { Overall Percentage: 68.14 }
%Thesis:
%Image Segmentation with Partial Convexity Prior Using Discrete Conformality Structures

{\bf The Chinese University of Hong Kong}, Hong Kong \hfill
{\em 2013 -- 2017} \\
BSc Mathematics; Minor in Economics %\hfill {cumulative GPA: 3.81/4.00} \\


%Minor in Linguistics \smallskip \\
%Member of Eta Kappa Nu \\
%Member of Upsilon Pi Epsilon \\


\end{rSection}

\begin{rSection}{Research and Professional Experiences}

{\bf Internship at Lawrence Berkeley National Laboratory} \hfill
{\em Summer 2023} 
\begin{itemize}
\item mentored by Dr Dmitriy Morozov, funded by the NSF MSGI program
\item built a neural network to learn the adsorption values of zeolite crystals from their topological features
\item partially verified a conjecture on the universality of a topological quantity on these datasets
\end{itemize}

{\bf Doctoral Thesis Research on Scale-Free Complexes} \hfill {\em Summer 2022 -- Present}
\begin{itemize}
\item ran simulations on a virtual machine to identify the topological properties of scale-free complexes
\item established theoretical guarantees and submitted the \href{https://arxiv.org/abs/2305.11259}{findings} to an academic journal
\item presented our findings in 3 workshops, and was invited to give talks at Chicago, Perdue and Seville
\item 
developing a GitHub repo with simulation codes and a Jupyter tutorial
\item mentored an undergraduate for 1 year and supervised her on literature review and programming
\end{itemize}


{\bf Doctoral Thesis Research on Small Density Vacuums} \hfill {\em Spring 2020 -- Summer 2022}
\begin{itemize}
\item developed a methodology to robustly identify small density vacuums in datasets and submitted a \href{https://arxiv.org/abs/2204.07821}{manuscript} to an academic journal
\item implemented the method and maintained a \href{https://github.com/c-siu/RDAD}{GitHub repo} with codes and a Jupyter tutorial
\item presented our findings in 7 workshops / conferences
\item made a \href{https://www.youtube.com/watch?v=X5c3_bbwQ7Q}{YouTube video} on our method
\item mentored an undergraduate for 2 year and supervised him on literature review and programming
\end{itemize}

{\bf Master Thesis Research on Image Segmentation} \hfill {\em Fall 2017 -- Summer 2019}
\begin{itemize}
\item developed an algorithm to segment convex objects in 2D images and implemented the method
\item published an \href{https://doi.org/10.1137/19M129718X}{article} in an academic journal
\end{itemize}

{\bf Other Research Activities}
\begin{itemize}
\item collaborated and coauthored journal articles with mathematicians and computer scientists on such topics as \href{https://doi.org/10.1007/s10915-021-01569-x}{computational geometry}, \href{https://ems.press/journals/jfg/articles/10964400}{fractal analysis} and \href{https://arxiv.org/abs/1610.00898}{knot theory (preprint)}
\end{itemize}
 
\end{rSection}
%\begin{rSection}{Carrier Objective}
% To work for an organization which provides me the opportunity to improve my skills and knowledge to grow along with the organization objective.
%\end{rSection}
%--------------------------------------------------------------------------------
%    Projects And Seminars
%-----------------------------------------------------------------------------------------------


\begin{rSection}{Teaching Experiences}

\textbf{Course Administrator} \hfill {\em Spring 2023}
\begin{itemize}
    \item coordinated test accommodations for students with time conflicts and students with disabilities in a 500-student course between the lecturers, the university administration and the students
    \item maintained the assignment submission system
\end{itemize}

\textbf{Teaching Assistant} \hfill {\em Fall 2022, Fall 2017 -- Spring 2019}
\begin{itemize}
    \item held weekly recitations
    \item developed recitation notes when needed
    \item graded assignments and exams
\end{itemize}

\textbf{Undergraduate Mentorship} \hfill {\em Spring 2020 -- Present}
\begin{itemize}
    \item led directed reading programs with 4 undergraduates
    \item coauthored with 2 mentees
\end{itemize}

\end{rSection}

\begin{rSection}{Select Awards and Honors}

\href{https://scholars.croucher.org.hk/scholars/siu-chun-yin}{\textbf{Croucher Scholarship for Doctoral Study}} \hfill {\em 2019/2020}\\
Annually, 9 -- 16 Hong Kong scholars pursuing an overseas doctoral degree in science, medicine or technology are selected.

\href{https://www.wfsfaa.gov.hk/sfo/seymf/en/index.htm}{\textbf{Sir Edward Youde Memorial Fellowship}} \hfill {\em 2017/2018}\\
Annually, 3 -- 5 fellows in Hong Kong are selected from the postgraudate students nominated by the heads of their institutions.

\end{rSection}

\begin{rSection}{Professional Services}
student representative of the Colloquium Committee, CAM, Cornell \hfill {\em Fall 23 -- Spring 24} \\
officer of SIAM Student Chapter, Cornell \hfill {\em Fall 22 -- Spring 24}
\end{rSection}


\begin{rSection}{Publications}

\emph{* denotes entries with alphabetically listed authors.}

\underline{C. Siu}, G. Samorodnitsky, C. Yu, and R. He. \href{https://arxiv.org/abs/2305.11259}{"The Asymptotics of the Expected Betti Numbers of Preferential Attachment Clique Complexes"}. Submitted.

\underline{C. Siu}, G. Samorodnitsky, C. Yu, and A. Yao. \href{https://arxiv.org/abs/2204.07821}{"Detection of Small Holes by the Scale-Invariant Robust Density-Aware Distance (RDAD) Filtration"}. Submitted.

* \underline{C. Siu}, and R. Strichartz. \href{https://ems.press/journals/jfg/articles/10964400}{"Geometry and Laplacian on Discrete Magic Carpets"}. \emph{Journal of Fractal Geometry}, 2023.

H. Law, \underline{C. Siu}, and R. Lui. \href{https://doi.org/10.1007/s10915-021-01569-x}{"Decomposition of Longitudinal Deformations via Beltrami Descriptors"}. \emph{Journal of Scientific Computing}, 2021.

\underline{C. Siu}, H.L. Chan, and R. Lui \href{https://doi.org/10.1137/19M129718X}{"Image Segmentation with Partial Convexity Shape Prior Using Discrete Conformality Structures"}. \emph{SIAM Journal on Imaging Sciences}, 2020.

* J. Li, and \underline{C. Siu}. \href{https://arxiv.org/abs/1610.00898}{"An Elementary Approach on Left-Orderability, Cables of Torus Knots and Dehn Surgery"}. Preprint.

%{\bf Topological Statistics} \hfill {\em spring 2019 -- present}\\
% develop persistence tools to highlight small high-density features in data sets\\
% under the supervision of Prof Gennady Samorodnitsky and Prof Christina Yu
% \iffalse
%\begin{itemize}
%    \item develop statistical foundation for inference of qualitative topological features of a data set
%    \item under the supervision of Prof Gennady Samorodnitsky and Prof Christina Yu
%\end{itemize}
%\fi

%{\bf HJB Equation and Time-Inconsistent Planning} \hfill {\em fall 2019 -- present}\\
%study the behavior of a present-biased planner, such as plan modification and procrastination\\
%bound the cost of present bias by the range of cost and the strength of bias\\
%supervised by Prof Alexander Vladimirsky
%\iffalse
%\begin{itemize}
%    \item study the behavior of a present-biased planner, such as plan modification and procrastination
%    \item bounded the cost of present bias by the range of cost and the strength of bias
%    \item supervised by Prof Alexander Vladimirsky
%\end{itemize}
%\fi

%{\bf Image Segmentation with Convexity Shape-Prior} \hfill {\em fall 2018 -- summer 2019}\\
%    developed an image segmentation algorithm with convexity guarantee using quasi-conformal geometry\\
%    joint work with Hei-Long Chan under the supervision of Prof Lok-Ming Lui\\
%    published at {SIAM Journal on Imaging Sciences}\\
%    doi: \href{https://doi.org/10.1137/19M129718X}{0.1137/19M129718X}
%    \iffalse
%\begin{itemize}
%    \item developed an image segmentation model that guarantees the convexity of the segmented region by quasi-conformal geometry
%    \item joint work with Hei-Long Chan under the supervision of Prof Lok-Ming Lui
%    \item published at SIAM Journal on Imaging Sciences 
%\end{itemize}
%\fi

%{\bf Analysis on Fractals} \hfill {\em summer 2016}\\
%proved the growth rate of ball sizes on the fractals IMC and studied the Laplacian on them\\
%joint work with Eric Goodman under the supervision of Prof Robert Strichartz\\
%accepted by Journal of Fractal Geometry 
%\iffalse
%\begin{itemize}
%    \item proved the growth rate of ball sizes on the fractal \emph{infinite magic carpet} and studied the Laplacian on them.
%    \item joint work with Eric Goodman under the supervision of Prof Robert Strichartz
%    \item submitted to Journal of Fractal Geometry 
%\end{itemize}
%\fi

\end{rSection}

%----------------------------------------------------------------------------------------
%	HONORS
%----------------------------------------------------------------------------------------



\begin{rSection}{Courses Taught}

MATH 1920 
Multivariable Calculus for Engineers, Cornell, head teaching assistant \hfill {\em Spring 23}\\
MATH 1920 
Multivariable Calculus for Engineers, Cornell, teaching assistant \hfill {\em Fall 22}\\
%Directed Reading Program on Applied Topology, Cornell, mentor \hfill {\em Fall 20, Spring 21, Spring 22}\\
MATH 2020 Advanced Calculus II, CUHK, teaching assistant \hfill {\em Spring 19}\\
MATH 4060 Complex Analysis, CUHK, teaching assistant \hfill {\em Fall 18}\\
MATH 2010 Advanced Calculus I, CUHK, teaching assistant \hfill {\em Spring 18}\\
MATH 1510 Calculus for Engineers, CUHK, teaching assistant \hfill {\em Spring 18}\\
MATH 1540 University Mathematics for Financial Studies, CUHK, teaching assistant \hfill {\em Fall 17}
\end{rSection}

\begin{rSection}{Undergraduate Mentees}

\href{https://www.linkedin.com/in/rongyi-caroline-he}{\textbf{Rongyi He}}, currently Cornell Master student \hfill {\em Summer 22 -- Summer 23}\\
\href{https://hoderle.in/}{\textbf{Luis Hoderlein}}, currently Yale PhD student \hfill {\em Spring 22 -- Summer 22}\\
\textbf{Tom Shi}, currently Cornell undergraduate student \hfill {\em Spring 22}\\
\textbf{Andrey Yao}, currently Madison PhD student \hfill {\em Fall 20 -- Spring 22}


\end{rSection}



\begin{rSection}{Additional Information}
\begin{tabular}{ @{} >{\bfseries}l @{\hspace{6ex}} l }
Natural languages & English, Chinese (Cantonese, Mandarin) \\
Programing & MATLAB, Python, Bash, R
\end{tabular}
\end{rSection}

\iffalse
\newpage
%----------------------------------------------------------------------------------------
%	TECHNICAL STRENGTHS SECTION
%----------------------------------------------------------------------------------------

\begin{rSection}{Technical Strengths}

\begin{tabular}{ @{} >{\bfseries}l @{\hspace{6ex}} l }
Modeling and Analysis \ & AutoCad, Revit, StaadPro \\
Software \& Tools & MS Office, Latex \\
\end{tabular}

\end{rSection}

%----------------------------------------------------------------------------------------
%	WORK EXPERIENCE SECTION
%----------------------------------------------------------------------------------------

\begin{rSection}{Work Experience}

\begin{rSubsection}{SJ Contracts, Pune}{June 2016}{Site Engineer}{}
\item On-site internship under this leading construction company. Learned and implemented various aspects such as quantity estimation, labour management and safety precautions.
\end{rSubsection}


\end{rSection}


%	EXAMPLE SECTION
%----------------------------------------------------------------------------------------

\begin{rSection}{Academic Achievements} 
 Runners up in B.G.Shirke Vidyarthi Competition for Innovative Project organized by Pune Construction Engineering Research Foundation in January 2018
\item Won First Prize in Model Making Competition Organized by Symbiosis Institute of Technology, Pune.
\end{rSection}

%----------------------------------------------------------------------------------------
% Extra Curricular
%----------------------------------------------------------------------------------------
\begin{rSection}{Extra-Cirrucular} \itemsep -3pt
\item Co-Organized “ Nirmitee 2017” - a National Symposium of Civil Department of MIT, Pune
\item Attended a workshop on Autodesk Revit at IIT Bombay in 2014.
\item Winner of Inter Departmental Football Competition 2015.
\item Member of the  Rotaract Club Of Pune Pride from 2014 to 2017.
\item Worked for a start-up company Named OUST as a Regional Marketing Manager
%\item Trained and disciplined in National Cadet Corps (NCC), IIT Kanpur for a year.
 %\item  Participated in Vijyoshi Camp 2012 organized at Indian Institute of Science, Bangalore.
 %\item Won 2nd position in Kho-Kho in Intramurals conducted by Physical Education Section, IIT Kanpur.
 %\item Pursued French as second language during secondary school from Grade 6 to Grade 10. Also participated in French Song Competition and French G.K. Quiz in Class 10th. %

\end{rSection}

\begin{rSection}{Personal Traits}
\item Highly motivated and eager to learn new things.
\item Strong motivational and leadership skills.
\item Ability to work as an individual as well as in group.
\end{rSection}
\fi

\end{document}
